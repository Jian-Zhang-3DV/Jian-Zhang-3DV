% !TEX TS-program = xelatex
% !TEX encoding = UTF-8 Unicode
% !Mode:: "TeX:UTF-8"

\documentclass{resume}
\usepackage{zh_CN-Adobefonts_external} % Simplified Chinese Support using external fonts (./fonts/zh_CN-Adobe/)
%\usepackage{zh_CN-Adobefonts_internal} % Simplified Chinese Support using system fonts
\usepackage{linespacing_fix} % disable extra space before next section
\usepackage{cite}
\usepackage{hyperref} % For clickable links
\usepackage{fontawesome} % Ensure FontAwesome is loaded

\begin{document}
\pagenumbering{gobble} % suppress displaying page number

\name{张舰}

\basicInfo{
  \email{zjrandomyeah@gmail.com} \textperiodcentered\ 
  \href{https://github.com/Jian-Zhang-3DV}{\faGithub\ Jian-Zhang-3DV} \textperiodcentered\ 
  \href{https://scholar.google.com/citations?user=qBNtBsAAAAAJ&hl=en&oi=sra}{\faGoogle\ Google Scholar} \textperiodcentered\
  \href{https://jian-zhang-3dv.github.io/Jian-Zhang-3DV/}{\faGlobe\ Homepage}
}
 
\section{\faGraduationCap\  教育背景}
\datedsubsection{\textbf{厦门大学}, 厦门}{2023年9月 -- 至今}
\textit{硕士研究生}
\datedsubsection{\textbf{南昌大学}, 南昌}{2019年9月 -- 2023年6月}
\textit{学士学位}, 人工智能专业

\section{\faFileTextO\ 学术成果}
% VLM-3R
\datedsubsection{\textbf{VLM-3R: Vision-Language Models Augmented with Instruction-Aligned 3D Reconstruction} (ArXiv)}{2025}
\textit{Authors:} \textbf{Jian Zhang}\textsuperscript{*}, Zhiwen Fan\textsuperscript{*}, Renjie Li, Junge Zhang, Runjin Chen, Hezhen Hu, Kevin Wang, Huaizhi Qu, Dilin Wang, Zhicheng Yan, Hongyu Xu, Justin Theiss, Tianlong Chen, Jiachen Li, Zhengzhong Tu, Zhangyang Wang, Rakesh Ranjan (*Equal Contribution)

VLM-3R:提出一个通过3D重建指令微调增强视觉-语言模型(VLM)的统一框架,可直接从单目视频处理3D信息。结合几何、相机视图token及超20万3D重建指令问答对,显著提升VLM空间理解和时空推理能力。

\href{https://arxiv.org/abs/2505.20279}{\faFileTextO\ Paper} \textperiodcentered\ \href{https://github.com/VITA-Group/VLM-3R}{\faCode\ Code} (\faStar\ 133, 开源于 2025-05-26) \textperiodcentered\ \href{https://vlm-3r.github.io/}{\faGlobe\ Project Page}

% Large Spatial Model
\datedsubsection{\textbf{Large Spatial Model: End-to-end Unposed Images to Semantic 3D} (NeurIPS)}{2024}
\textit{Authors:} \textbf{Jian Zhang}\textsuperscript{*}, Zhiwen Fan\textsuperscript{*}, Wenyan Cong, Peihao Wang, Renjie Li, Kairun Wen, Shijie Zhou, Achuta Kadambi, Zhangyang Wang, Danfei Xu, Boris Ivanovic, Marco Pavone, Yue Wang (*Equal Contribution)

LSM:提出一个端到端框架,从无姿态RGB图像实时重建含几何、外观和语义的3D辐射场。通过Transformer、跨视角注意力和多模态融合,有效整合2D图像特征与语义至3D空间,首次实现实时语义3D重建,在新视角合成和开放词汇3D分割上表现优越。

\href{https://arxiv.org/abs/2410.18956}{\faFileTextO\ Paper} \textperiodcentered\ \href{https://github.com/NVlabs/LSM}{\faCode\ Code} (\faStar\ 186, 开源于 2024-12-22, Citations: 13) \textperiodcentered\ \href{https://largespatialmodel.github.io/}{\faGlobe\ Project Page}

% InstantSplat
\datedsubsection{\textbf{InstantSplat: Sparse-view Gaussian Splatting in Seconds} (ArXiv)}{2024}
\textit{Authors:} Zhiwen Fan\textsuperscript{*}, Kairun Wen\textsuperscript{*}, Wenyan Cong\textsuperscript{*}, Kevin Wang, \textbf{Jian Zhang}, Xinghao Ding, Danfei Xu, Boris Ivanovic, Marco Pavone, Georgios Pavlakos, Zhangyang Wang, Yue Wang (*Equal Contribution)

InstantSplat:提出一种基于自监督学习和神经渲染的快速3D场景重建方法。通过高效优化3D场景表示(高斯溅射)和相机位姿,从稀疏视角2D图像快速生成高质量3D模型。

\href{https://arxiv.org/abs/2403.20309}{\faFileTextO\ Paper} \textperiodcentered\ \href{https://github.com/NVlabs/InstantSplat}{\faCode\ Code} (\faStar\ 1.4k, 开源于 2024-08-01, Citations: 92) \textperiodcentered\ \href{https://instantsplat.github.io/}{\faGlobe\ Project Page}

% DynamicVerse
\datedsubsection{\textbf{DynamicVerse: Physically-Aware Multimodal Modeling for Dynamic 4D Worlds} (Preprint)}{2025}
\textit{Authors:} Kairun Wen\textsuperscript{*}, Yuzhi Huang\textsuperscript{*}, Runyu Chen, Hui Zheng, Yunlong Lin, Panwang Pan, Chenxin Li, Wenyan Cong, \textbf{Jian Zhang}, Junbin Lu, Chenguo Lin, Dilin Wang, Zhicheng Yan, Hongyu Xu, Justin Theiss, Yue Huang, Xinghao Ding, Rakesh Ranjan, Zhiwen Fan (*Equal Contribution)

DynamicVerse:提出一个物理感知的多模态4D建模框架,理解动态世界的3D结构、物理运动及文本语义。通过大模型处理信息,将长视频转为4D多模态格式,并提供超10万视频和80万标注掩码的大规模数据集,赋能具身智能体。

\href{https://dynamic-verse.github.io/}{\faGlobe\ Project Page} (论文与代码即将发布)

\section{\faTrophy\  获奖经历}
\datedsubsection{\textbf{南昌大学特等奖学金}}{2023年}
\datedsubsection{\textbf{南昌大学优秀毕业生}}{2023年}

%% Reference
%\newpage
%\bibliographystyle{IEEETran}
%\bibliography{mycite}
\end{document} 